\section{Framework}
\label{sec:framework}
Ono što čini Clojure atraktivnim je veliki spektar šablona i modularnih biblioteka. Dovoljno je univerzalan da radi na gotovo svakom JVM-u, dok je dinamičan do te mere da nudi širok spektar funkcionalnosti.


Neki od poznatijih framework-a za Clojure su:\cite{frameworks}

\begin{itemize}
            \item Luminus
            \item Hoplon
            \item ClojureHomePage
\end{itemize}

\subsection{Luminus}
\label{subsec:luminus}
Luminus mikro-framework je zasnovan na skupu laganih(malih) biblioteka, čiji je cilj da obezbedi robustnu, skalabilnu i jednostavnu platformu. Lumius funkcioniše kao vrsta šablonskong sistema, obezbeđujući ugrađene razvojne sisteme i neke
podrazumevane module za jumpstart razvoj.\cite{luminus}

% Ne treba apostrof na S i nisam sigurna da razumem ovu rečenicu. Možda malo preformulisati?
% S obzirom na to, postoji nešto što se može reći za kreiranje purpose-built modula za specifičnu
% implementaciju, i da se odluči da se to ne koristi radi bržeg razvoja, može imati svoje troškove u budućnosti.

S obzirom da ima mogućnost za jumpstart razvoj, psotoji moduli za kreiranje specifične implementacije. U slučaju
da se takvi moduli ne koriste da bi se ubrzao razvoj, moguće je da se pojave dodatni troškovi u budućnosti.

Postoji mogućnost da se projekat započne veoma brzo, a istovremeno onemoguće moduli od kojih nema trenutne koristi. Ovo rezultuje niskom potrošnjom resursa i visokom efikasnošću implementacije i to zahvaljujući tome što je modularnost unapred definisana.\cite{frameworks}

\subsection{Hoplon}
\label{subsec:hoplon}
Hoplon je skup Clojure i ClojureScript biblioteka, povezanih zajedno sa Boot build alatom, koji ujedinjuju neke idiosinkrazije veb platforme i predstavljaju interesantan način dizajniranja i izrade veb stranica sa jednom stranicom.\cite{hoplon}


Hoplon pruža kompajler za izradu frontend veb aplikacija, i  sadrži sledeće biblioteke od kojih zavisi:\cite{hoplon_wiki}
    \begin{enumerate}
        \item Javelin -
        Biblioteka za protok podataka za kontrolu stanja klijenta. Hoplon se čvrsto
        integrira sa Javelinom kako bi reaktivno vezao DOM elemente na grafikonu koji se nalazi u pozadini
        Javelinove ćelije.
        \item Castra -
        Potpuno opremljena RPC biblioteka za Clojure i ClojureScript, obezbeđujuć i
        okruženje servera. (opciono)
    \end{enumerate}

\subsection{ClojureHomePage}
\label{subsec:clojurehomepage}
ClojureHomePage je jedinstven framework, koji se u potpunosti fokusira na kreiranje veb stranice koja koristi Clojure za front i backend.
Podržava širok raspon veb zaglavlja, varijabli okruženja i ruta, i sposoban je generirati prilično složen HTML i CSS. 
Pored toga, njegova podrška za SKL funkcije znači da, dok se frontend generiše, pozadina se takođe generiše na veoma čvrsto integrisan i moćan način.\cite{frameworks}
