\section{Instalacija}
\label{sec:instalacija}
Clojure pruža skup alata komandne linije koji se mogu koristiti za pokretanje Clojure REPL-a, upotrebu Java i Clojure biblioteka i pokretanje Clojure programa.
\subsection{Linux}
\label{subsec:linux}
\begin{enumerate}
            \item Obezbediti da su instalirani paketi: \texttt{curl}, \texttt{rlwrap}, i \texttt{java}.
            Provera verzija pomenutih paketa se vrši putem komandi:
            \begin{tcolorbox}[colback=green!5!white,colframe=green!5!white,fontupper=\ttfamily]
            \$ curl -V \newline
            \$ rlwrap -v \newline
            \$ java -version
            \end{tcolorbox}
            Instalacija:
            \begin{tcolorbox}[colback=green!5!white,colframe=green!5!white,fontupper=\ttfamily]
            \$ sudo apt install curl \newline
            \$ sudo apt install rlwrap
            \end{tcolorbox}
            Ako nije instalirana Java:
            \begin{tcolorbox}[colback=green!5!white,colframe=green!5!white,fontupper=\ttfamily]
            \$ sudo apt install default-jre \newline
            \$ sudo apt install openjdk-11-jdk
            \end{tcolorbox}
            \item Preuzeti i instalirati pomoću linux skripta za instalaciju, koji će kreirati fajlove /usr/local/bin/clj, /usr/local/bin/clojure, and /usr/local/lib/clojure:
\begin{tcolorbox}[colback=green!5!white,colframe=green!5!white,fontupper=\ttfamily]

\$ curl -O https://download.clojure.org/\\install/linux-install-1.10.0.442.sh

\$ chmod +x linux-install-1.10.0.442.sh

\$ sudo ./linux-install-1.10.0.442.sh
\end{tcolorbox}

\end{enumerate}

\subsection{Windows}
\label{subsec:windows}
Trenutno postoji samo alfa verzija Clojure-a za Windows. Najpre se sa \href{https://github.com/clojure/tools.deps.alpha/wiki/clj-on-Windows}{link-a} preuzme poslednja verzija instalacionog fajla. Nakon pokretanja instalacije, biće ponuđene 3 moguće lokacije za instalaciju:\newline
\begin{tcolorbox}[colback=green!5!white,colframe=green!5!white,fontupper=\ttfamily]
Possible install locations:\newline
  1) \path{\\Drive\Home\Documents\WindowsPowerShell\Modules}\\
  2) \path{C:\Program Files\WindowsPowerShell\Modules}\\
  3) \path{C:\WINDOWS\system32\WindowsPowerShell\v1.0\Modules\ }\\
Enter number of preferred install location: 
\end{tcolorbox}
Treba imati u vidu da pri izboru 1. opcije nisu potrebne admin privilegije, ali se kreira dodatni fajl u \path{\Documents}, dok za 2. i 3. opciju je potrebno imati admin privilegije.

\subsection{Pokretanje}
\label{subsec:pokretanje}
Nakon preuzimanja i instalacije potrebnih alata, REPL se pokreće pomoću komande \texttt{clj}:
\begin{tcolorbox}[colback=green!5!white,colframe=green!5!white,fontupper=\ttfamily]
\$ clj\\
Clojure 1.10.0\\
user=>
\end{tcolorbox}

Prilikom ulaska u REPL, moguće je kucati Clojure izraze i pokretati ih pritiskom na Enter.
Postoji veliki broj Clojure i Java biblioteka koje nude razne funkcionalnosti. Često korišćena biblioteka je \href{https://github.com/clj-time/clj-time}{clj-time} koja radi sa datumima i vremenom.


Da bi se koristila ova biblioteka potrebno je napraviti \textbf{deps.edn} fajl za deklarisanje zavisnosti:
\begin{tcolorbox}[colback=green!5!white,colframe=green!5!white,fontupper=\ttfamily]
\{:deps


\hspace*{5mm}\{clj-time \{:mvn/version ''0.14.2''\}\}\}
\end{tcolorbox}
Za pisanje programa, potrebno je napraviti novi direktorijum i kopirati \textbf{deps.edn} fajl u odgovarajući direktorijum.

Komanda \texttt{clj} automatski traži izvorne fajlove u \textbf{src} direktorijumu pa je potrebno fajl sa ekstenzijom \underline{\textit{.clj}} sačuvati na putanji \path{\src\program.clj} i pokrenuti:

\begin{tcolorbox}[colback=green!5!white,colframe=green!5!white,fontupper=\ttfamily]
\$ clj -m program
\end{tcolorbox}
%\subsubsection{Clojure online}
 
%\subsubsection{Lokalno građenje}
%Clojure je moguće graditi lokalno (neophodni su paketi \texttt{git}, \texttt{java} i \texttt{maven}):
%\begin{tcolorbox}[colback=green!5!white,colframe=green!5!white,fontupper=\ttfamily]
%    \$ git clone https://github.com/clojure/clojure.git\newline
%    \$ cd clojure\newline
%    \$ mvn -Plocal -Dmaven.test.skip=true package
%\end{tcolorbox}
%Pokretanje REPL pomoću lokalnog jar-a:
%\begin{tcolorbox}[colback=green!5!white,colframe=green!5!white,fontupper=\ttfamily]
%    java -jar clojure.jar
%\end{tcolorbox}

Pored instalacije na lokalnom računaru, REPL je moguće koristiti i u web pregledaču pomoću servisa \href{https://repl.it/languages/clojure}{repl.it}. Ova web stranica omogućuje korišćenje više nezavisnih REPL interfejsa, kao i povezivanje sa Github nalogom.