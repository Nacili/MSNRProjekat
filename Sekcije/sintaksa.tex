\section{Sintaksna struktura}
\label{sec:sintaksnastruktura}

Kako je Clojure dijalekat programskog jezika Lisp, i sintaksa ta dva jezika je slična. Ono što je karakteristično za ovakve jezike jeste njihova sintaksna \emph{uniformnost}.

\subsection{Literali i operatori}
\label{subsec:literalioperatori}

\lstinputlisting[language=Clojure, caption={Literali},frame=single, label=literali]{Kodovi/literali.clj}

\textbf{Literali} su najjednostavniji izrazi - to su oni izrazi koji se evaluiraju u sami sebe. Neki od osnovnih literala u programskom jeziku Clojure su:

\begin{itemize}
    \item Celi brojevi
    \item Brojevi u pokretnom zarezu
    \item Razlomljeni brojevi
    \item Niske
    \item Ključne reči
\end{itemize}

Brojevi se predstavljaju na način koji je standardan u programskim jezicima, niske se predstavljaju sa dvostrukim navodnicima (jednostruki nisu dozvoljeni), dok su ključne reči obeležene sa dvotačkom ispred reči (ali reč je navedena \emph{bez} navodnika).
Ključne reči se koriste za referisanje, na primer prilikom dohvatanja elemenata mape.

Literali se koriste zajedno uz pomoć \textbf{operatora} - ugrađenih funkcija. Korišćenje operatora (i drugih definisanih funkcija) se obavlja na sledeći način:

\begin{verbatim}
(operator operand_1 operand_2)
\end{verbatim}

Opisan je binarni, ali se na analogan način koristi i n-arni operator. Primetimo da su zagrade \emph{obavezne}, kao i da se operandi razdvajaju blanko karakterom (zapeta se takođe smatra blanko karakterom). Clojure svoju uniformnost održava onemogućavajući operatorima da se navode na drugačiji način. Na primer, uobičajena infiksna notacija aritmetičkih operatora nije omogućena u programskom jeziku Clojure. 

Neki od operatora su dati u narednom listingu:

\lstinputlisting[language=Clojure, caption={Aritmetički, relacioni i logički operatori},frame=single, label=operatori]{Kodovi/operatori.clj}

Operator \texttt{/} se različito ponaša u zavisnosti od operanada: ukoliko su oba operanda celi brojevi, onda je rezultat izračunavanja tog izraza razlomljeni broj ili ceo broj (ako je prvi operand deljiv drugim). Ukoliko je neki od operanada broj u pokretnom zarezu, onda je i rezultat broj u pokretnom zarezu.

Dodeljivanje imena vrednostima se vrši pomoću operatora \texttt{def}:

\begin{verbatim}
(def ime_konstante operand_1)
\end{verbatim}

U drugim programskim jezicima, vrednosti se dodeljuju promenljivama. To nije slučaj u Clojure-u, zbog njegove imutabilne prirode.

\subsection{Kontrola toka}
\label{subsec:kontrolatoka}

U Clojure-u nisu izostali mehanizmi kontrole toka. Za naredbu grananja, koristi se operator \texttt{if}:

\lstinputlisting[language=Clojure, caption={Operatori \texttt{if} i \texttt{do}},frame=single, label=operator_if]{Kodovi/operator_if.clj}

Operator \texttt{if}, ukoliko je vrednost prvog operanda \texttt{true} vraća vrednost drugog operanda; inače, vraća vrednost trećeg operanda. Else grana u okviru operatora \texttt{if} nije obavezna.

Treba napomenuti da se sve vrednosti tretiraju kao tačne, osim literala \texttt{nil} i \texttt{false}. Operator \texttt{and} vraća poslednju tačnu vrednost ukoliko su svi operandi tačni; inače vraća prvu netačnu vrednost. Analogno, operator \texttt{or} vraća poslednju netačnu vrednost ukoliko su svi operandi netačni; inače, vraća se vrednost prvog tačnog operanda.

Operator \texttt{do} omogućava da se izvrši više operacija u nekoj ili obe grane. Povratna vrednost ovog operatora je vrednost poslednjeg operanda.

\subsection{Definisanje funkcija}
\label{subsec:definisanjefja}

Korišćenje funkcija je identično korišćenju operatora. Njihovo definisanje se vrši pomoću operatora \texttt{defn} sledeći način:

\lstinputlisting[language=Clojure, caption={Definisanje funkcija sa tačnim i proizvoljnim brojem parametara},frame=single, label=funkcije]{Kodovi/def_funkcija.clj}

Prvi operand je ime funkcije (navodi se \emph{bez} navodnika), zatim (opciono) postavljanje deskripcije koja se može dohvatiti operatorom \texttt{doc}, treći operand je vektor koji sadrži argumente funkcije i, na kraju, povratna vrednost funkcije.

Primetimo da je moguće definisati funkciju tako da uzima proizvoljan broj parametara; to se postiže korišćenjem simbola \texttt{\&} koji će poslate parametre smestiti u listu.

\subsection{Strukture podataka}
\label{strukturepodataka}

U Clojure-u je dostupno korišćenje tradicionalnih struktura podataka kao što su liste, vektori, mape i skupovi.  Treba napomenuti da elementi ovih struktura podataka ne moraju biti istog tipa.

Liste su definisane na sledeći način:

\begin{verbatim}
'(element_1 element_2 element_3)
\end{verbatim}

Na sličan način se definišu i vektori, ali se koriste simboli \texttt{[} i \texttt{]}. Operatorom \texttt{nth} se može dohvatiti n-ti element liste, dok se ista stvar za vektore postiže korišćenjem operatora \texttt{get}. Dodavanje novih elemenata u listu ili vektor se vrši pomoću operatora \texttt{conj}; novi element će biti dodat na početak liste, odnosno na kraj vektora.

Mape se definišu ili putem operatora \texttt{hash-map} ili uz pomoć vitičastih zagrada:

\begin{verbatim}
{:prva_kljucna_rec vrednost_1 :druga_kljucna_rec vrednost_2}
\end{verbatim}

Pomenuti \texttt{get} operator će vratiti vrednost navedene ključne reči.

Skupovi su vektori čiji su svi elementi jedinstveni. Njihova definicija je:

\begin{verbatim}
#{element_1 element_2 element_3}
\end{verbatim}

Naknadna pojavljivanja istih elemenata u okviru istog skupa biće ignorisana. Za proveru pripadnosti neke vrednosti skupu koristi se operator \texttt{contains?}.