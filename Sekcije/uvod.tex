\section{Uvod}
\label{sec:uvod}

\epigraph{"Bolje je imati 100 funkcija koje operišu nad jednom strukturom podataka, nego 10 funkcija koje operišu nad 10 različitih struktura podataka."}{\textit{ --- Alan J. Perlis}}

 Kao savremeni dijalekt Lisp programskog jezika, čiji makro-sistem poseduje, Clojure posmatra kod kao podatke nepromenljive strukture. Uspešno kombinuje pristupačnost i interaktivan razvoj, kakav se može susresti kod skriptnog jezika, sa efikasnom i robustnom infrastrukturom višenitnog programiranja.
 
 Uprkos činjenici da je kao jezik kompajliran, on uspeva da očuva potpunu dinamičnost i time omogući da svaka osobenost podržana od strane Clojure-a bude podržana i za vreme izvršavanja. Njegov osnovni interfejs za programiranje, REPL, nas oslobađa ograničenja u smislu kompajliranja i pokretanja izvršnog koda kao jedinih opcija i daje slobodu interaktivnog pisanja programa.

Kao mlad jezik, Clojure je ređe korišćen, ako ne i slabije poznat u programerskim krugovima, uprkos velikim mogućnostima koje pruža. Često je okarakterisan, od strane svojih pristalica, kao zabavan jezik koji dopušta drugačiji pogled na samu logiku programiranja.

Ono što čini Clojure veoma omiljenim i pristupačnim, jeste veliki dijapazon šablona i modularnih biblioteka. Posedujući visok stepen unapred definisane modularnosti, on omogućava da se projekat započne odmah, a istovremeno onemoguće moduli od kojih nema trenutne koristi, čime rezultuje niskom potrošnjom resursa i visokom efikasnošću implementacije.

U ovom radu, čitaocu će najpre biti predstavljena osnovna svojstva jezika i pojašnjenja njegove sintakse, a zatim će, vođen uputstvom za instalaciju i primerom koda biti u mogućnosti da se i sam okuša u programiranju.