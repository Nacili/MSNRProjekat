\section{Zaključak}
\label{sec:zakljucak}

Već nakon prvog upoznavanja s Clojure-om, primećuje se da je kao jezik jednostavan, elegantan i kako zahteva manje kodiranja nego ostali jezici. Obezbeđuje bezbednu manipulaciju konkurentnim podacima i to na visokom nivou, podršku za unit i generativne testove, kao i za standardnu biblioteku za ceo JVM ekosistem. Njegova sintaksa je visoko-proširiva pomoću makroa i literala čitača. Podržava čitav niz apstrakcija koje potpomažu umnogome skalabilan kod i olakšavaju njegovo refaktorisanje.

Uprkos svojim poželjnim osobinama, Clojure poseduje i određena ograničenja. Nije memorijski efikasan, ne može se koristiti za osetljive servise koji rade u realnom vremenu i njegove poruke o grešci mogu biti teške za dešifrovanje. Inicijalizacija okruženja može biti spora u odnosu na konkurenciju, a i nedostatak eksplicitnih i statičkih  tipova otežava testiranje.

Kako je sam jezik još uvek u razvoju, postaje kompatibilan sa sve većim brojem platformi, te se može  očekivati proširenje njegovih oblasti primene. U industriji ga koriste Apple, Atlassian, Netflix, Walmart i vladina tela kao što je NASA, a sve češće se koristi i u sferama poput muzike, videa, bioinformatike. Kako dobija sve više pristalica i postaje sve popularniji programski jezik, očekuje se da Clojure postane dominantan u odnosu na  svoje funkcionalne srodnike.
